\chapter{Introduction}

The most common use of information technology is to store and retrieve data, be it text, images, video, or audio files. As the amount of data generated by several processes increases as time goes on, storage systems must evolve to guarantee reliable access to the data, as well as fast and efficient retrieval. Data is typically stored into \textbf{databases} (\textbf{DBs}), which are housed in a permanent memory.

The technology on which permanent memory is based uses magnetic \textbf{disks}, containing a set of platters that rotate at relatively slow speeds (compared to CPU speed), which can be interacted with by using heads attached to moving arms. Each platter has on both surfaces a set of rings, called \textbf{tracks}, which, except for the innermost and outermost ones, are used to store information. Each track is subdivided into \textbf{sectors} of the same size, which correspond to the smallest unit of transfer allowed by the hardware. Typical sector sizes are 512 bytes, 1 KB, 2 KB, or 4 KB. There are from 500 to 1000 sectors per track, and about 100K tracks per surface of a single platter.

The \textbf{access time} needed to read a section of the disk is given by the seek time (needed to move the head), the rotational delay (given by the spinning of the disk itself), and the transfer time (needed to read/write the data). These operations take several milliseconds to be completed, which are definitely slower than any operation relative to the \textbf{main memory} (\textbf{RAM}), taking only a few nanoseconds in total. 

Despite this disparity, disks are still today the preferred technology to store data. Main memory is, in fact, volatile: once the machine stops receiving electricity powering it on, any information on the RAM is lost forever. On the other hand, disks provide reliable storage: the information written on them can be retrieved even if the machine is turned off and on. A newer technology, called \textbf{solid state storage}, and, in particular, \textbf{flash memory}, has risen in popularity in the last years. It provides the reliability of disks and much faster operations, although they still haven't become the new standard since they tend to be expensive.